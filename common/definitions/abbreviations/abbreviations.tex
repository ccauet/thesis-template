%!TEX root = ../../../main.tex

%===============================================================================
%% Overline for antiparticles

\newcommand*{\ovE}[1]{%
  \overline{{#1}}%
  %\smash{\overline{#1}}%
  %\overbar{#1}
}%

%===============================================================================
%% Shortcuts for convenient math symbol definition
\newcommand{\newmathsymbol}[2]{
  \newcommand{#1}{\ensuremath{#2}\xspace}
}
\newcommand{\renewmathsymbol}[2]{
  \renewcommand{#1}{\ensuremath{#2}\xspace}
}

% language
%!TEX root = ../../../main.tex

%% Common language abbreviations
\newcommand{\ie}          {i.e.\xspace}
\newcommand{\eg}          {e.g.\xspace}
\newcommand{\etc}         {etc.\xspace}
% general
%!TEX root = ../../../main.tex

%% Sample categories
\newcommand{\catll}              {LL\xspace}
\newcommand{\catdd}              {DD\xspace}
\newcommand{\catlllg}            {long track \KS\xspace}
\newcommand{\catddlg}            {downstream track \KS\xspace}

%% Tables
% Table empty cell dash
\newcommand{\tbdash}             {---}
% math
%!TEX root = ../../../main.tex

%===============================================================================
%% Math abbreviations

%% Equation typesetting
% Equation ending with period and comma
\newmathsymbol{\eqcm}{\,\text{,}\xspace}
\newmathsymbol{\eqpd}{\,\text{.}\xspace}
% textspace in equations
\newmathsymbol{\eqspace}{\quad}

% software
%!TEX root = ../../../main.tex

\newcommand{\ROOT}           {\textsc{ROOT}\xspace}
\newcommand{\RooFit}         {\textsc{RooFit}\xspace}
\newcommand{\Minuit}         {\textsc{Minuit}\xspace}
\newcommand{\MinuitTwo}      {\textsc{Minuit2}\xspace}
\newcommand{\Hesse}          {\textsc{Hesse}\xspace}
\newcommand{\Migrad}         {\textsc{Migrad}\xspace}
\newcommand{\Minos}          {\textsc{Minos}\xspace}

\newcommand{\Gaudi}          {\textsc{Gaudi}\xspace}
\newcommand{\Brunel}         {\textsc{Brunel}\xspace}
\newcommand{\Moore}          {\textsc{Moore}\xspace}
\newcommand{\DaVinci}        {\textsc{DaVinci}\xspace}
\newcommand{\Davinci}        {\textsc{DaVinci}\xspace}
\newcommand{\Gauss}          {\textsc{Gauss}\xspace}
\newcommand{\Boole}          {\textsc{Boole}\xspace}
\newcommand{\Pythia}         {\textsc{Pythia}\xspace}
\newcommand{\PythiaSix}      {\textsc{Pythia6}\xspace}
\newcommand{\PythiaEight}    {\textsc{Pythia8}\xspace}
\newcommand{\EvtGen}         {\textsc{EvtGen}\xspace}
\newcommand{\Photos}         {\textsc{Photos}\xspace}
\newcommand{\Herwig}         {\textsc{Herwig}\xspace}
\newcommand{\Herwigpp}       {\textsc{Herwig++}\xspace}
\newcommand{\Sherpa}         {\textsc{Sherpa}\xspace}
\newcommand{\GeantFour}      {\textsc{Geant4}\xspace}

\newcommand{\FlavourTagging} {\textsc{FlavourTagging}\xspace}

\SaveVerb{tmp_recofourteen}=Reco14=
\newcommand{\RecoFourteen}   {\texttt{Reco14}\xspace}

\newmathsymbol{\dtf}         {\text{\acs*{DTF}}_{\text{\acs*{PV}}}^{\jpsi\KS}\xspace}
\newmathsymbol{\dtfpv}       {\text{\acs*{DTF}}_{\text{\acs*{PV}}}\xspace}

% quarks
%!TEX root = ../../../main.tex

\def\quark     {\ensuremath{\mathrm{q}}\xspace}
\def\quarkbar  {\ensuremath{\overline \quark}\xspace}
\def\qqbar     {\ensuremath{\quark\quarkbar}\xspace}
\def\uquark    {\ensuremath{\mathrm{u}}\xspace}
\def\uquarkbar {\ensuremath{\overline \uquark}\xspace}
\def\uubar     {\ensuremath{\uquark\uquarkbar}\xspace}
\def\dquark    {\ensuremath{\mathrm{d}}\xspace}
\def\dquarkbar {\ensuremath{\overline \dquark}\xspace}
\def\ddbar     {\ensuremath{\dquark\dquarkbar}\xspace}
\def\squark    {\ensuremath{\mathrm{s}}\xspace}
\def\squarkbar {\ensuremath{\overline \squark}\xspace}
\def\ssbar     {\ensuremath{\squark\squarkbar}\xspace}
\def\cquark    {\ensuremath{\mathrm{c}}\xspace}
\def\cquarkbar {\ensuremath{\overline \cquark}\xspace}
\def\ccbar     {\ensuremath{\cquark\cquarkbar}\xspace}
\def\bquark    {\ensuremath{\mathrm{b}}\xspace}
\def\bquarkbar {\ensuremath{\overline \bquark}\xspace}
\def\bbbar     {\ensuremath{\bquark\bquarkbar}\xspace}
\def\tquark    {\ensuremath{\mathrm{t}}\xspace}
\def\tquarkbar {\ensuremath{\overline \tquark}\xspace}
\def\ttbar     {\ensuremath{\tquark\tquarkbar}\xspace}

% leptons
%!TEX root = ../../main.tex

\def\electron   {\ensuremath{\mathrm{e}}\xspace}
\def\ep         {\ensuremath{\electron^+}\xspace}
\def\en         {\ensuremath{\electron^-}\xspace}
\def\epm        {\ensuremath{\electron^\pm}\xspace}
\def\epem       {\ensuremath{\electron^+\electron^-}\xspace}

\def\muon       {\ensuremath{\mu}\xspace}
\def\mup        {\ensuremath{\muon^+}\xspace}
\def\mun        {\ensuremath{\muon^-}\xspace}
\def\mupm       {\ensuremath{\muon^\pm}\xspace}
\def\mumu       {\ensuremath{\muon^+\muon^-}\xspace}

\def\tauz       {\ensuremath{\tau}\xspace}
\def\taup       {\ensuremath{\tau^+}\xspace}
\def\taum       {\ensuremath{\tau^-}\xspace}
\def\taupm      {\ensuremath{\tau^\pm}\xspace}
\def\tautau     {\ensuremath{\tau^+\tau^-}\xspace}

\def\neutrino   {\ensuremath{\nu}\xspace}
\def\neutrinobar{\ensuremath{\overline{\neutrino}}\xspace}

\def\neutrinoe  {\ensuremath{\neutrino_\electron}\xspace}
\def\neutrinoeb {\ensuremath{\neutrinob_\electron}\xspace}

\def\neutrinom  {\ensuremath{\neutrino_\muon}\xspace}
\def\neutrinomb {\ensuremath{\neutrinob_\muon}\xspace}

\def\neutrinot  {\ensuremath{\neutrino_\tauz}\xspace}
\def\neutrinotb {\ensuremath{\neutrinob_\tauz}\xspace}

\def\neutrinol  {\ensuremath{\neutrino_\ell}\xspace}
\def\neutrinolb {\ensuremath{\neutrinob_\ell}\xspace}

% hadrons (not specfically mesons/baryons)
%!TEX root = ../../../main.tex

%===============================================================================
%% Mesons

%% General beauty hadron
% math symbol
\newmathsymbol{\hb}{h_\bquark}
% text
\newcommand{\bhadron}{$\bquark$ hadron\xspace}
\newcommand{\bhadrons}{$\bquark$ hadrons\xspace}

% mesons
%!TEX root = ../../../main.tex

%===============================================================================
%% Mesons

%% Pions
\newmathsymbol{\pion}        {\pi}
\newmathsymbol{\piz}         {\pion^0}
\newmathsymbol{\pip}         {\pion^+}
\newmathsymbol{\pim}         {\pion^-}
\newmathsymbol{\pipm}        {\pion^\pm}
\newmathsymbol{\pimp}        {\pion^\mp}
\newmathsymbol{\pipi}        {\pip\pim}

%% All them Kaons
\newmathsymbol{\kaon}        {K}
\newmathsymbol{\Kbar}        {\kern 0.2em\ovE{\kern -0.2em \kaon}{}}
\newmathsymbol{\Kz}          {\kaon^0}
\newmathsymbol{\Kzbar}       {\Kb^0}
\newmathsymbol{\Kp}          {\kaon^+}
\newmathsymbol{\Km}          {\kaon^-}
\newmathsymbol{\Kpm}         {\kaon^\pm}
\newmathsymbol{\Kmp}         {\kaon^\mp}
\newmathsymbol{\KK}          {\Kp\Km}
\newmathsymbol{\KS}          {\kaon^0_{\text{S}}} 
\newmathsymbol{\KL}          {\kaon^0_{\text{L}}} 

\newmathsymbol{\Kstar}       {\kaon^\ast}
\newmathsymbol{\Kstarbar}    {\Kb^\ast}
\newmathsymbol{\Kstarz}      {\kaon^{\ast 0}}
\newmathsymbol{\Kstarzbar}   {\Kb^{\ast 0}}
\newmathsymbol{\Kstarp}      {\kaon^{\ast +}}
\newmathsymbol{\Kstarm}      {\kaon^{\ast -}}
\newmathsymbol{\Kstarpm}     {\kaon^{\ast \pm}}
\newmathsymbol{\Kstarmp}     {\kaon^{\ast \mp}}

%% Quarkonia
% Kerning for Jpsi is very font-dependent. Maybe it's necessary to provide 
% individual definitions for each font…
\newmathsymbol{\jpsi}        {J\kern-.25ex/\kern-.1ex\psi}
\newmathsymbol{\Jpsi}        {\jpsi}
\newmathsymbol{\psitwos}     {\psi{(2S)}}

%% Beautiful B mesons
% Could not find out who defined \B before
\renewmathsymbol{\B}         {B} 
\newmathsymbol{\Bbar}        {\kern 0.16em\ovE{\kern -0.16em \B}{}}
\newmathsymbol{\Bz}          {\B^0}
\newmathsymbol{\Bzbar}       {\Bbar^0}
\newmathsymbol{\Bu}          {\B^+}
\newmathsymbol{\Bubar}       {\B^-}
\newmathsymbol{\Bp}          {\Bu}
\newmathsymbol{\Bm}          {\Bubar}
\newmathsymbol{\Bpm}         {\B^\pm}
\newmathsymbol{\Bmp}         {\B^\mp}
\newmathsymbol{\Bd}          {\Bz}
\newmathsymbol{\Bdbar}       {\Bzbar}
\newmathsymbol{\Bs}          {\B^0_\squark}
\newmathsymbol{\Bsbar}       {\Bbar^0_\squark}
\newmathsymbol{\Bc}          {\B_\cquark^+}
\newmathsymbol{\Bcp}         {\B_\cquark^+}
\newmathsymbol{\Bcm}         {\B_\cquark^-}
\newmathsymbol{\Bcpm}        {\B_\cquark^\pm}
% B(s): Bs or B0
\newmathsymbol{\Bsd}         {\B^0_{(\squark)}}
\newmathsymbol{\Bsdbar}      {\Bbar^0_{(\squark)}}


% baryons
%!TEX root = ../../../main.tex

%===============================================================================
%% Baryons

%% Nucleons
\newmathsymbol{\proton}     {p\xspace}
\newmathsymbol{\protonbar}  {\ovE{\proton}\xspace}
\newmathsymbol{\pp}         {\proton\proton\xspace}
\newmathsymbol{\ppbar}      {\proton\protonbar\xspace}

\newmathsymbol{\neutron}    {n\xspace}
\newmathsymbol{\neutronbar} {\ovE{\neutron}\xspace}

%% Strangest Lambdas
\newmathsymbol{\Lambdap}    {\mathit{\Lambda}}
\newmathsymbol{\Lambdaz}    {\Lambdap^0\xspace}
\newmathsymbol{\Lambdazbar} {\ovE{\Lambdap}^0\xspace}
\newmathsymbol{\Lambdabar}  {\kern 0.1em\ovE{\kern -0.1em\Lambdap}{}\xspace}
\newmathsymbol{\Lambdab}    {\Lambdap^0_\bquark\xspace}
\newmathsymbol{\Lambdabbar} {\Lambdabar^0_\bquark\xspace}
\newmathsymbol{\Lambdac}    {\Lambdap^+_\cquark\xspace}
\newmathsymbol{\Lambdacbar} {\Lambdabar^-_\cquark\xspace}

% \def\Deltares   {\ensuremath{\Delta}\xspace}
% \def\Deltaresbar{\ensuremath{\overline \Deltares}\xspace}
% \def\Xires {\ensuremath{\Xi}\xspace}
% \def\Xiresbar{\ensuremath{\overline \Xires}\xspace}
% \def\Sigmares {\ensuremath{\Sigma}\xspace}
% \def\Sigmaresbar{\ensuremath{\overline \Sigmares}\xspace}
% \def\Omegares {\ensuremath{\Omega^-}\xspace}
% \def\Omegaresbar{\ensuremath{\overline{\POmega}^+}\xspace}


% decays
%!TEX root = ../../../main.tex

\newcommand{\BF}          {{\ensuremath{\cal B}\xspace}}
\newcommand{\BR}          {\BF}

\renewcommand{\to}        {\ensuremath{\rightarrow}\xspace}
\newcommand{\decay}[2]    {\ensuremath{#1\!\to #2}\xspace}

%% Common decays 
\newcommand{\BdToJpsiKS}      {\decay{\Bd}{\jpsi\KS}}
\newcommand{\BsToJpsiKS}      {\decay{\Bs}{\jpsi\KS}}
\newcommand{\BdToPsiTwoSKS}   {\decay{\Bd}{\psi(2S)\KS}}
\newcommand{\BdToJpsiX}       {\decay{\Bd}{\jpsi X}}
\newcommand{\LbToJpsiLambda}  {\decay{\Lb}{\jpsi\L}}
\newcommand{\BuToJpsiK}       {\decay{\Bu}{\jpsi\Kp}}
\newcommand{\BuToJpsiKcc}     {\decay{\Bpm}{\jpsi\Kpm}}
\newcommand{\BuToDPi}         {\decay{\Bu}{\Dz\pip}}
\newcommand{\BdToDPi}         {\decay{\Bd}{\Dp\pim}}
\newcommand{\JpsiToMuMu}      {\decay{\jpsi}{\mumu}}
\newcommand{\KSToPiPi}        {\decay{\KS}{\pip\pim}}
\newcommand{\BdToDstmunu}     {\decay{\Bd}{\Dstarm\mup\neum}}
\newcommand{\BsToDsPi}        {\decay{\Bs}{\Ds\pim}}
\newcommand{\inclJPsi}        {inclusive \jpsi}

%% Decays to be used in headings with bold, sans, italic math font (makes sure 
%  all characters use the proper font)
\newcommand{\BdToJpsiKSbfsf}  {\ensuremath{\mathbfsfit{
  \decay{B^{\mathsf{0}}}{\jpsi K_\text{S}^{\mathsf{0}}}   
}}\xspace}
\newcommand{\BsToJpsiKSbfsf}  {\ensuremath{\mathbfsfit{
  \decay{B_s^{\mathsf{0}}}{\jpsi K_\text{S}^{\mathsf{0}}}   
}}\xspace}



% units
%!TEX root = ../../../main.tex

%===============================================================================
%% Units definitions

\DeclareSIUnit\clight{\ensuremath{c}}

\DeclareSIUnit\rad{\radian}
\DeclareSIUnit\mrad{\milli\rad} 

\DeclareSIUnit\micron{\micro\metre} 

\DeclareSIUnit\nanobarn{\nano\barn} 
\DeclareSIUnit\picobarn{\pico\barn} 
\DeclareSIUnit\femtobarn{\femto\barn} 
\DeclareSIUnit\attobarn{\atto\barn}
\DeclareSIUnit\nb{\nano\barn} 
\DeclareSIUnit\pb{\pico\barn} 
\DeclareSIUnit\fb{\femto\barn} 
\DeclareSIUnit\ab{\atto\barn} 
\DeclareSIUnit\zb{\zepto\barn} 
\DeclareSIUnit\yb{\yocto\barn}

\DeclareSIUnit[per-mode=symbol]\keV{\kilo\eV}
\DeclareSIUnit[per-mode=symbol]\MeV{\mega\eV}
\DeclareSIUnit[per-mode=symbol]\GeV{\giga\eV}
\DeclareSIUnit[per-mode=symbol]\TeV{\tera\eV}

\DeclareSIUnit[per-mode=symbol]\eVc{\eV\per\clight}
\DeclareSIUnit[per-mode=symbol]\keVc{\kilo\eV\per\clight}
\DeclareSIUnit[per-mode=symbol]\MeVc{\mega\eV\per\clight}
\DeclareSIUnit[per-mode=symbol]\GeVc{\giga\eV\per\clight}
\DeclareSIUnit[per-mode=symbol]\TeVc{\tera\eV\per\clight}

\DeclareSIUnit[per-mode=symbol]\eVcc{\eV\per\square\clight}
\DeclareSIUnit[per-mode=symbol]\keVcc{\kilo\eV\per\square\clight}
\DeclareSIUnit[per-mode=symbol]\MeVcc{\mega\eV\per\square\clight}
\DeclareSIUnit[per-mode=symbol]\GeVcc{\giga\eV\per\square\clight}
\DeclareSIUnit[per-mode=symbol]\TeVcc{\tera\eV\per\square\clight}

% chemical compounds and reactions
%!TEX root = ../../../main.tex

\newcommand{\water}{\ch{H2O}\xspace}
\newcommand{\fluorocarbon}{\ch{CF10}\xspace}
\newcommand{\fluorocarbonfour}{\ch{C4F10}\xspace}

