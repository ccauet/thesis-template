%!TEX root = ../../../main.tex

%===============================================================================
%% Overline for antiparticles

\newcommand*{\ovE}[1]{%
  \overline{{#1}}%
  %\smash{\overline{#1}}%
  %\overbar{#1}
}%

%===============================================================================
%% Shortcuts for convenient math symbol definition
\newcommand{\newmathsymbol}[2]{
  \newcommand{#1}{\ensuremath{#2}\xspace}
}
\newcommand{\renewmathsymbol}[2]{
  \renewcommand{#1}{\ensuremath{#2}\xspace}
}

% language
%!TEX root = ../../../main.tex

%% Experiments
\newcommand{\ie}          {i.e.\xspace}
\newcommand{\eg}          {e.g.\xspace}
\newcommand{\etc}         {etc.\xspace}
% general
%!TEX root = ../../main.tex

% CKM, CP violation
\def\eps   {\ensuremath{\varepsilon}\xspace}
\def\epsK  {\ensuremath{\varepsilon_K}\xspace}
\def\epsB  {\ensuremath{\varepsilon_B}\xspace}
\def\epsp  {\ensuremath{\varepsilon^\prime_K}\xspace}

\def\CP                {\ensuremath{C\!P}\xspace}
\def\CPT               {\ensuremath{C\!PT}\xspace}

\def\rhobar {\ensuremath{\overline \rho}\xspace}
\def\etabar {\ensuremath{\overline \eta}\xspace}

\def\Vud  {\ensuremath{|V_{\uquark\dquark}|}\xspace}
\def\Vcd  {\ensuremath{|V_{\cquark\dquark}|}\xspace}
\def\Vtd  {\ensuremath{|V_{\tquark\dquark}|}\xspace}
\def\Vus  {\ensuremath{|V_{\uquark\squark}|}\xspace}
\def\Vcs  {\ensuremath{|V_{\cquark\squark}|}\xspace}
\def\Vts  {\ensuremath{|V_{\tquark\squark}|}\xspace}
\def\Vub  {\ensuremath{|V_{\uquark\bquark}|}\xspace}
\def\Vcb  {\ensuremath{|V_{\cquark\bquark}|}\xspace}
\def\Vtb  {\ensuremath{|V_{\tquark\bquark}|}\xspace}

\newcommand{\SJpsiKS}{\ensuremath{S_{\jpsi\KS}}\xspace}
\newcommand{\CJpsiKS}{\ensuremath{C_{\jpsi\KS}}\xspace}
\newcommand{\SJpsiKSblind}{\ensuremath{S_{\jpsi\KS}^{\text{blind}}}\xspace}
\newcommand{\CJpsiKSblind}{\ensuremath{C_{\jpsi\KS}^{\text{blind}}}\xspace}

\newcommand{\sintwobeta}{\ensuremath{\sin 2\beta}\xspace}

% Oscillations
\newcommand{\dm}{\ensuremath{\Delta m}\xspace}
\newcommand{\dms}{\ensuremath{\Delta m_{\squark}}\xspace}
\newcommand{\dmd}{\ensuremath{\Delta m_{\dquark}}\xspace}
\newcommand{\DG}{\ensuremath{\Delta\Gamma}\xspace}
\newcommand{\DGs}{\ensuremath{\Delta\Gamma_{\squark}}\xspace}
\newcommand{\DGd}{\ensuremath{\Delta\Gamma_{\dquark}}\xspace}
\newcommand{\Gs}{\ensuremath{\Gamma_{\squark}}\xspace}
\newcommand{\Gd}{\ensuremath{\Gamma_{\dquark}}\xspace}

% QM
\newcommand{\bra}[1]{\ensuremath{\langle #1|}}             % {a}
\newcommand{\ket}[1]{\ensuremath{|#1\rangle}}              % {b}
\newcommand{\braket}[2]{\ensuremath{\langle #1|#2\rangle}} % {a}{b}

% Uncertainties
\newcommand{\stat}{\ensuremath{\text{\,(stat)}}\xspace}
\newcommand{\syst}{\ensuremath{\text{\,(syst)}}\xspace}

% software
%!TEX root = ../../../main.tex

\newcommand{\Gaudi}          {\textsc{Gaudi}\xspace}
\newcommand{\Brunel}         {\textsc{Brunel}\xspace}
\newcommand{\Moore}          {\textsc{Moore}\xspace}
\newcommand{\DaVinci}        {\textsc{DaVinci}\xspace}
\newcommand{\Davinci}        {\textsc{DaVinci}\xspace}
\newcommand{\Gauss}          {\textsc{Gauss}\xspace}
\newcommand{\Boole}          {\textsc{Boole}\xspace}
\newcommand{\Pythia}         {\textsc{Pythia}\xspace}
\newcommand{\PythiaSix}      {\textsc{Pythia6}\xspace}
\newcommand{\PythiaEight}    {\textsc{Pythia8}\xspace}
\newcommand{\EvtGen}         {\textsc{EvtGen}\xspace}
\newcommand{\Photos}         {\textsc{Photos}\xspace}
\newcommand{\Herwig}         {\textsc{Herwig}\xspace}
\newcommand{\Herwigpp}       {\textsc{Herwig++}\xspace}
\newcommand{\Sherpa}         {\textsc{Sherpa}\xspace}
\newcommand{\GeantFour}      {\textsc{Geant4}\xspace}


% quarks
%!TEX root = ../../../main.tex

\def\quark     {\ensuremath{\mathrm{q}}\xspace}
\def\quarkbar  {\ensuremath{\overline \quark}\xspace}
\def\qqbar     {\ensuremath{\quark\quarkbar}\xspace}
\def\uquark    {\ensuremath{\mathrm{u}}\xspace}
\def\uquarkbar {\ensuremath{\overline \uquark}\xspace}
\def\uubar     {\ensuremath{\uquark\uquarkbar}\xspace}
\def\dquark    {\ensuremath{\mathrm{d}}\xspace}
\def\dquarkbar {\ensuremath{\overline \dquark}\xspace}
\def\ddbar     {\ensuremath{\dquark\dquarkbar}\xspace}
\def\squark    {\ensuremath{\mathrm{s}}\xspace}
\def\squarkbar {\ensuremath{\overline \squark}\xspace}
\def\ssbar     {\ensuremath{\squark\squarkbar}\xspace}
\def\cquark    {\ensuremath{\mathrm{c}}\xspace}
\def\cquarkbar {\ensuremath{\overline \cquark}\xspace}
\def\ccbar     {\ensuremath{\cquark\cquarkbar}\xspace}
\def\bquark    {\ensuremath{\mathrm{b}}\xspace}
\def\bquarkbar {\ensuremath{\overline \bquark}\xspace}
\def\bbbar     {\ensuremath{\bquark\bquarkbar}\xspace}
\def\tquark    {\ensuremath{\mathrm{t}}\xspace}
\def\tquarkbar {\ensuremath{\overline \tquark}\xspace}
\def\ttbar     {\ensuremath{\tquark\tquarkbar}\xspace}

% leptons
%!TEX root = ../../../main.tex

%===============================================================================
%% Leptons

\newmathsymbol{\lepton}      {l}
\newmathsymbol{\lepp}        {\lepton^+}
\newmathsymbol{\lepm}        {\lepton^-}
\newmathsymbol{\leppm}       {\lepton^\pm}
\newmathsymbol{\leplem}      {\lepp\lepm}

\newmathsymbol{\electron}    {e}
\newmathsymbol{\elp}         {\electron^+}
\newmathsymbol{\elm}         {\electron^-}
\newmathsymbol{\elpm}        {\electron^\pm}
\newmathsymbol{\elel}        {\elp\elm}

\newmathsymbol{\muon}        {\mu}
\newmathsymbol{\mup}         {\muon^+}
\newmathsymbol{\mum}         {\muon^-}
\newmathsymbol{\mupm}        {\muon^\pm}
\newmathsymbol{\mumu}        {\mup\mum}

\newmathsymbol{\tauon}       {\tau}
\newmathsymbol{\taup}        {\tauon^+}
\newmathsymbol{\taum}        {\tauon^-}
\newmathsymbol{\taupm}       {\tauon^\pm}
\newmathsymbol{\tautau}      {\taup\taum}

\newmathsymbol{\neutrino}    {\nu}
\newmathsymbol{\neutrinobar} {\ovE \neutrino}

\newmathsymbol{\nuel}        {\neutrino_\electron}
\newmathsymbol{\nuelbar}     {\neutrinobar_\electron}

\newmathsymbol{\numu}        {\neutrino_\muon}
\newmathsymbol{\numubar}     {\neutrinobar_\muon}

\newmathsymbol{\nutau}       {\neutrino_\tauon}
\newmathsymbol{\nutaubar}    {\neutrinobar_\tauon}

\newmathsymbol{\nulep}       {\neutrino_\lepton}
\newmathsymbol{\nulepbar}    {\neutrinobar_\lepton}

% hadrons (not specfically mesons/baryons)
%!TEX root = ../../../main.tex

%===============================================================================
%% Mesons

%% General beauty hadron
\newmathsymbol{\hb}{h_\bquark}

% mesons
%!TEX root = ../../../main.tex

%===============================================================================
%% Mesons

%% Pions
\newmathsymbol{\pion}        {\pi}
\newmathsymbol{\piz}         {\pion^0}
\newmathsymbol{\pip}         {\pion^+}
\newmathsymbol{\pim}         {\pion^-}
\newmathsymbol{\pipm}        {\pion^\pm}
\newmathsymbol{\pimp}        {\pion^\mp}
\newmathsymbol{\pipi}        {\pip\pim}

%% All them Kaons
\newmathsymbol{\kaon}        {K}
\newmathsymbol{\Kaon}        {\kaon}
\newmathsymbol{\Kbar}        {\kern 0.2em\ovE{\kern -0.2em \kaon}{}}
\newmathsymbol{\Kz}          {\kaon^0}
\newmathsymbol{\Kzbar}       {\Kbar^0}
\newmathsymbol{\Kp}          {\kaon^+}
\newmathsymbol{\Km}          {\kaon^-}
\newmathsymbol{\Kpm}         {\kaon^\pm}
\newmathsymbol{\Kmp}         {\kaon^\mp}
\newmathsymbol{\KK}          {\Kp\Km}
\newmathsymbol{\KS}          {\kaon^0_{\text{S}}} 
\newmathsymbol{\KL}          {\kaon^0_{\text{L}}} 

\newmathsymbol{\Kstar}       {\kaon^\ast}
\newmathsymbol{\Kstarbar}    {\Kbar^\ast}
\newmathsymbol{\Kstarz}      {\kaon^{\ast 0}}
\newmathsymbol{\Kstarzbar}   {\Kbar^{\ast 0}}
\newmathsymbol{\Kstarp}      {\kaon^{\ast +}}
\newmathsymbol{\Kstarm}      {\kaon^{\ast -}}
\newmathsymbol{\Kstarpm}     {\kaon^{\ast \pm}}
\newmathsymbol{\Kstarmp}     {\kaon^{\ast \mp}}

%% Quarkonia
% Kerning for Jpsi is very font-dependent. Maybe it's necessary to provide 
% individual definitions for each font…
\newmathsymbol{\jpsi}        {J\kern-.25ex/\kern-.1ex\psi}
\newmathsymbol{\Jpsi}        {\jpsi}
\newmathsymbol{\psitwos}     {\psi{(2S)}}

\newmathsymbol{\YOneS}  {\Upsilon(1S)}
\newmathsymbol{\YTwoS}  {\Upsilon(2S)}
\newmathsymbol{\YThreeS}{\Upsilon(3S)}
\newmathsymbol{\YFourS} {\Upsilon(4S)}
\newmathsymbol{\YFiveS} {\Upsilon(5S)}

%% Charming D mesons
\newmathsymbol{\D}           {D}
\newmathsymbol{\Dbar}        {\kern 0.2em\ovE{\kern -0.2em \D}{}}
\renewmathsymbol{\Dz}        {\D^0}
\newmathsymbol{\Dzbar}       {\Dbar^0}
\newmathsymbol{\Dp}          {\D^+}
\newmathsymbol{\Dm}          {\D^-}
\newmathsymbol{\Dpm}         {\D^\pm}
\newmathsymbol{\Dmp}         {\D^\mp}

\newmathsymbol{\Dstar}       {\D^\ast}
\newmathsymbol{\Dstarbar}    {\Dbar^\ast}
\newmathsymbol{\Dstarz}      {\D^{\ast 0}}
\newmathsymbol{\Dstarzbar}   {\Kb^{\ast 0}}
\newmathsymbol{\Dstarp}      {\D^{\ast +}}
\newmathsymbol{\Dstarm}      {\D^{\ast -}}
\newmathsymbol{\Dstarpm}     {\D^{\ast \pm}}
\newmathsymbol{\Dstarmp}     {\D^{\ast \mp}}

\newmathsymbol{\Ds}          {D_\squark}
\newmathsymbol{\Dsp}         {\Ds^+}
\newmathsymbol{\Dsm}         {\Ds^-}
\newmathsymbol{\Dspm}        {\Ds^\pm}

\newmathsymbol{\Dsstar}      {\Ds^\ast}
\newmathsymbol{\Dsstarp}     {\Ds^{\ast +}}
\newmathsymbol{\Dsstarm}     {\Ds^{\ast -}}
\newmathsymbol{\Dsstarpm}    {\Ds^{\ast \pm}}
\newmathsymbol{\Dsstarmp}    {\Ds^{\ast \mp}}

%% Beautiful B mesons
% Could not find out who defined \B before
\renewmathsymbol{\B}         {B} 
\newmathsymbol{\Bbar}        {\kern 0.16em\ovE{\kern -0.16em \B}{}}
\newmathsymbol{\Bz}          {\B^0}
\newmathsymbol{\Bzbar}       {\Bbar^0}
\newmathsymbol{\Bu}          {\B^+}
\newmathsymbol{\Bubar}       {\B^-}
\newmathsymbol{\Bp}          {\Bu}
\newmathsymbol{\Bm}          {\Bubar}
\newmathsymbol{\Bpm}         {\B^\pm}
\newmathsymbol{\Bmp}         {\B^\mp}
\newmathsymbol{\Bd}          {\Bz}
\newmathsymbol{\Bdbar}       {\Bzbar}
\newmathsymbol{\Bdstar}      {{\Bd}^{\ast}}
\newmathsymbol{\Bs}          {\B^0_\squark}
\newmathsymbol{\Bsbar}       {\Bbar^0_\squark}
\newmathsymbol{\Bc}          {\B_\cquark^+}
\newmathsymbol{\Bcp}         {\B_\cquark^+}
\newmathsymbol{\Bcm}         {\B_\cquark^-}
\newmathsymbol{\Bcpm}        {\B_\cquark^\pm}
% B(s): Bs or B0
\newmathsymbol{\Bsd}         {\B^0_{(\squark)}}
\newmathsymbol{\Bsdbar}      {\Bbar^0_{(\squark)}}

%% Mesons in text
% b meson
\newcommand{\bmeson}{$\bquark$ meson\xspace}
\newcommand{\bmesons}{$\bquark$ mesons\xspace}
% B meson
\newcommand{\Bmeson}{$\B$ meson\xspace}
\newcommand{\Bmesons}{$\B$ mesons\xspace}

%% Mesons to be used in headings with bold, sans, italic math font (makes sure 
%  all characters use the proper font)
\newmathsymbol{\Bdbfsf}      {\mathbfsfit{\B^{\mathsf{0}}}}

%% Mesons with hyperref-compatible text alternative
\newcommand{\BdHyperref}     {\texorpdfstring{\Bd}{B0}\xspace}


% baryons
%!TEX root = ../../../main.tex

%===============================================================================
%% Baryons

%% Nucleons
\newcommand{\proton}      {\ensuremath{p}\xspace}
\newcommand{\protonbar}   {\ensuremath{\ovE \proton}\xspace}
\newcommand{\pp}          {\ensuremath{\proton\proton}\xspace}

\newcommand{\neutron}     {\ensuremath{n}\xspace}
\newcommand{\neutronbar}  {\ensuremath{\ovE \neutron}\xspace}

%% Strangest Lambdas
\newcommand{\Lambdabar}   {\ensuremath{\kern 0.1em\ovE{\kern -0.1em\Lambda}{}}\xspace}
\newcommand{\Lambdab}     {\ensuremath{\Lambda^0_\bquark}\xspace}
\newcommand{\Lambdabbar}  {\ensuremath{\Lambdabar^0_\bquark}\xspace}
\newcommand{\Lambdac}     {\ensuremath{\Lambda^+_\cquark}\xspace}
\newcommand{\Lambdacbar}  {\ensuremath{\Lambdabar^-_\cquark}\xspace}

% \def\Deltares   {\ensuremath{\Delta}\xspace}
% \def\Deltaresbar{\ensuremath{\overline \Deltares}\xspace}
% \def\Xires {\ensuremath{\Xi}\xspace}
% \def\Xiresbar{\ensuremath{\overline \Xires}\xspace}
% \def\Sigmares {\ensuremath{\Sigma}\xspace}
% \def\Sigmaresbar{\ensuremath{\overline \Sigmares}\xspace}
% \def\Omegares {\ensuremath{\Omega^-}\xspace}
% \def\Omegaresbar{\ensuremath{\overline{\POmega}^+}\xspace}


% decays
%!TEX root = ../../main.tex

\def\BF         {{\ensuremath{\cal B}\xspace}}
\def\BR         {\BF}

\def\to               {\ensuremath{\rightarrow}\xspace}
\newcommand{\decay}[2]{\ensuremath{#1\!\to #2}\xspace}         % {\Pa}{\Pb \Pc}


\newcommand{\BdToJpsiKS}{\decay{\Bd}{\jpsi\KS}}
\newcommand{\BsToJpsiKS}{\decay{\Bs}{\jpsi\KS}}
\newcommand{\BdToPsiTwoSKS}{\decay{\Bd}{\psi(2S)\KS}}
\newcommand{\BdToJpsiX}{\decay{\Bd}{\jpsi X}}
\newcommand{\LbToJpsiLambda}{\decay{\Lb}{\jpsi\L}}
\newcommand{\BuToJpsiK}{\decay{\Bu}{\jpsi\Kp}}
\newcommand{\BuToJpsiKcc}{\decay{\Bpm}{\jpsi\Kpm}}
\newcommand{\BuToDPi}{\decay{\Bu}{\Dz\pip}}
\newcommand{\BdToDPi}{\decay{\Bd}{\Dp\pim}}
\newcommand{\JpsiToMuMu}{\decay{\jpsi}{\mumu}}
\newcommand{\KSToPiPi}{\decay{\KS}{\pip\pim}}
\newcommand{\BdToDstmunu}{\decay{\Bd}{\Dstarm\mup\neum}}
\newcommand{\BsToDsPi}{\decay{\Bs}{\Ds\pim}}
\newcommand{\inclJPsi}{inclusive \jpsi}


% units
%!TEX root = ../../../main.tex

%===============================================================================
%% Units definitions

\DeclareSIUnit\clight{\ensuremath{c}}

\DeclareSIUnit\rad{\radian}
\DeclareSIUnit\mrad{\milli\rad} 

\DeclareSIUnit\mm{\milli\metre} 
\DeclareSIUnit\micron{\micro\metre} 
\DeclareSIUnit\um{\micro\metre} 

\DeclareSIUnit\millibarn{\milli\barn} 
\DeclareSIUnit\microbarn{\micro\barn} 
\DeclareSIUnit\nanobarn{\nano\barn} 
\DeclareSIUnit\picobarn{\pico\barn} 
\DeclareSIUnit\femtobarn{\femto\barn} 
\DeclareSIUnit\attobarn{\atto\barn}
\DeclareSIUnit\nb{\nano\barn} 
\DeclareSIUnit\pb{\pico\barn} 
\DeclareSIUnit\fb{\femto\barn} 
\DeclareSIUnit\ab{\atto\barn} 
\DeclareSIUnit\zb{\zepto\barn} 
\DeclareSIUnit\yb{\yocto\barn}

\DeclareSIUnit\picosecond{\pico\second} 
\DeclareSIUnit\ps{\pico\second} 
\DeclareSIUnit\ns{\nano\second} 

\DeclareSIUnit\lumi{\per\square\centi\metre\per\second}

% Slightly optimising the kerning for eV
\DeclareSIUnit\electronvolt{\electron\kern-.1em \volt}

\DeclareSIUnit[per-mode=symbol]\keV{\kilo\eV}
\DeclareSIUnit[per-mode=symbol]\MeV{\mega\eV}
\DeclareSIUnit[per-mode=symbol]\GeV{\giga\eV}
\DeclareSIUnit[per-mode=symbol]\TeV{\tera\eV}

\DeclareSIUnit[per-mode=symbol]\eVc{\eV\per\clight}
\DeclareSIUnit[per-mode=symbol]\keVc{\kilo\eV\per\clight}
\DeclareSIUnit[per-mode=symbol]\MeVc{\mega\eV\per\clight}
\DeclareSIUnit[per-mode=symbol]\GeVc{\giga\eV\per\clight}
\DeclareSIUnit[per-mode=symbol]\TeVc{\tera\eV\per\clight}

\DeclareSIUnit[per-mode=symbol]\eVcc{\eV\per\square\clight}
\DeclareSIUnit[per-mode=symbol]\keVcc{\kilo\eV\per\square\clight}
\DeclareSIUnit[per-mode=symbol]\MeVcc{\mega\eV\per\square\clight}
\DeclareSIUnit[per-mode=symbol]\GeVcc{\giga\eV\per\square\clight}
\DeclareSIUnit[per-mode=symbol]\TeVcc{\tera\eV\per\square\clight}

\DeclareSIUnit[number-unit-product = \,]{\permille}{\textperthousand}

% Tm as new unit; no way to auto-optimise kerning here
\DeclareSIUnit\Tm{\tesla\kern-0.07em\meter}

% chemical compounds and reactions
%!TEX root = ../../../main.tex

\newcommand{\water}{\ch{H2O}\xspace}
\newcommand{\fluorocarbon}{\ch{CF10}\xspace}
\newcommand{\fluorocarbonfour}{\ch{C4F10}\xspace}

