%!TEX root = ../../../main.tex

%===============================================================================
%% Abbreviations used in High Energy Physics

%% Experiments
\newcommand{\RunOne}{Run I\xspace}
\newcommand{\RunTwo}{Run II\xspace}

\newcommand{\BFactory}{$\B$ factory\xspace}
\newcommand{\BFactories}{$\B$ factories\xspace}
\newcommand{\Babar}{\textsc{BaBar}\xspace}
\newcommand{\Belle}{Belle\xspace}
\newcommand{\BelleTwo}{Belle II\xspace}

%% Analysis techniques
\newcommand{\sweight}{sWeight\xspace}
\newcommand{\sweights}{sWeights\xspace}
\newcommand{\sWeights}{\sweights}
\newcommand{\sweighted}{sWeighted\xspace}
\newcommand{\sPlot}{sPlot\xspace}
\newcommand{\splot}{\sPlot}

%% Number of degrees of freedom
\newmathsymbol{\ndf}{\text{ndf}}

%% Signal and Background
\newmathsymbol{\Sig}{\text{Sig}}
\newmathsymbol{\Bkg}{\text{Bkg}}

%% HEP quantities
\newmathsymbol{\sqrts}{\sqrt{s}}

%% chi2/ndf
\newmathsymbol{\chisq}{\chi^2}
\newcommand{\redchisq}[1]{\ensuremath{\chisq_{#1}/\ndf}\xspace}
\newmathsymbol{\chisqndf}{\redchisq{}}

%% Probability Density Function
\newcommand{\Prob}[2]{\ensuremath{\mathcal{P}_{#1}^{#2}}}
\newcommand{\ProbArg}[3]{\ensuremath{\Prob{#1}{#2}\kern-.2ex \left(#3\right)}}

%% Normalisation
\newcommand{\Norm}[2]{\ensuremath{\mathcal{N}_{#1}^{#2}}}

%% Resolution
\newcommand{\Resolution}[2]{\ensuremath{\mathcal{R}_{#1}^{#2}}}

%% Likelihood function
\newcommand{\Likelihood}[2]{\ensuremath{\mathcal{L}_{#1}^{#2}}}

%% Hamiltonian
\newcommand{\Hamiltonian}[2]{\ensuremath{\mathcal{H}_{#1}^{#2}}}

%% Asymmetries
\newcommand{\Asym}[2]{\ensuremath{\mathcal{A}_{#1}^{#2}}}

%% Lagrangian
\newcommand{\Lagrangian}[2]{\ensuremath{\mathcal{L}_{#1}^{#2}}}
\newmathsymbol{\hc}{\text{h.c.}}

%% Delta log-likelihood
\newmathsymbol{\DLL}{\Delta\text{LL}}
\newmathsymbol{\DLLKpi}{\DLL_{\kaon\pion}}
\newmathsymbol{\DLLppi}{\DLL_{\proton\pion}}
\newmathsymbol{\DLLmupi}{\DLL_{\muon\pion}}
\newmathsymbol{\DLLepi}{\DLL_{\electron\pion}}
\newmathsymbol{\DLLlpi}{\DLL_{\lepton\pion}}

%% NNPID variables
\newmathsymbol{\ProbNN}{\Prob{\text{NN}}{}}
\newmathsymbol{\ProbNNK}{\ProbArg{\text{NN}}{}{\kaon}}
\newmathsymbol{\ProbNNpi}{\ProbArg{\text{NN}}{}{\pion}}
\newmathsymbol{\ProbNNp}{\ProbArg{\text{NN}}{}{\proton}}
\newmathsymbol{\ProbNNmu}{\ProbArg{\text{NN}}{}{\muon}}
\newmathsymbol{\ProbNNe}{\ProbArg{\text{NN}}{}{\electron}}

%% Efficiencies
\newmathsymbol{\SigEff}{\varepsilon_{\Sig}}
\newmathsymbol{\BkgEff}{\varepsilon_{\Bkg}}
\newmathsymbol{\BkgRej}{1-\varepsilon_{\Bkg}}

%% Uncertainties
\newmathsymbol{\stat}{\text{stat}}
\newmathsymbol{\syst}{\text{syst}}
\newmathsymbol{\statp}{\text{\,(stat)}}
\newmathsymbol{\systp}{\text{\,(syst)}}

%% CKM matrix
\newmathsymbol{\VCKM}             {V_{\text{\acs*{CKM}}}}

%% CKM matrix elements
\newmathsymbol{\Vud}              {V_{\uquark\dquark}}
\newmathsymbol{\Vcd}              {V_{\cquark\dquark}}
\newmathsymbol{\Vtd}              {V_{\tquark\dquark}}
\newmathsymbol{\Vus}              {V_{\uquark\squark}}
\newmathsymbol{\Vcs}              {V_{\cquark\squark}}
\newmathsymbol{\Vts}              {V_{\tquark\squark}}
\newmathsymbol{\Vub}              {V_{\uquark\bquark}}
\newmathsymbol{\Vcb}              {V_{\cquark\bquark}}
\newmathsymbol{\Vtb}              {V_{\tquark\bquark}}

%% B meson oscillation parameters
\newmathsymbol{\DM}       {\Delta m}
\newmathsymbol{\DMs}      {\DM_{\squark}}
\newmathsymbol{\DMd}      {\DM_{\dquark}}
\newmathsymbol{\DG}       {\Delta\Gamma}
\newmathsymbol{\DGs}      {\DG_{\squark}}
\newmathsymbol{\DGd}      {\DG_{\dquark}}
\newmathsymbol{\Gs}       {\Gamma_{\squark}}
\newmathsymbol{\Gd}       {\Gamma_{\dquark}}

\newmathsymbol{\MixingAsymmetry}{\Asym{\text{Mix}}{}}

%% CP Violation
\newmathsymbol{\CP}               {C\kern-.25ex P}
\newmathsymbol{\CPbfsf}           {\mathbfsfit{\CP}}
\newmathsymbol{\CPT}              {C\kern-.25ex P\kern-.05ex T}
\newmathsymbol{\CPAsymmetry}      {\Asym{\CP}{}}

%% CP observables
\newmathsymbol{\SJpsiKS}      {S_{\jpsi\KS}}
\newmathsymbol{\CJpsiKS}      {C_{\jpsi\KS}}
\newmathsymbol{\sintwobeta}   {\sin(2\beta)}

%% Group theory
\newcommand{\group}[2]{\ensuremath{\text{#1}\left(#2\right)}}

%% Bra-Ket notation
\newcommand{\bra}[1]            {\ensuremath{\bigl\langle #1 \bigr\vert}}
\newcommand{\ket}[1]            {\ensuremath{\bigl\vert   #1 \bigr\rangle}}
\newcommand{\braket}[2]         {\ensuremath{\bigl\langle #1 \big\vert #2 \bigr\rangle}}
\newcommand{\matrixelement}[3]  {\ensuremath{\bigl\langle #1 \big\vert #2 \big\vert #3 \bigr\rangle}}



