%!TEX root = ../../../main.tex

%===============================================================================
%% Math abbreviations

%% Equation typesetting
% Equation ending with period and comma
\newmathsymbol{\eqcm}{\,\text{,}\xspace}
\newmathsymbol{\eqpd}{\,\text{.}\xspace}
% textspace in equations
\newmathsymbol{\eqspace}{\quad}
\newmathsymbol{\eqthinspace}{\enskip}
% text "and" in equations
\newmathsymbol{\eqand}{\eqthinspace\text{and}\eqthinspace}

%% pipe in conditional statements
\newmathsymbol{\given}{\,\middle\vert\,}

%% dx/Dx
\newcommand{\dif}{\mathop{}\!\mathrm{d}}
\newcommand{\Dif}{\mathop{}\!\mathrm{D}}

%% Exponential function
\newcommand{\exponential}[1]{\ensuremath{\mathrm{e}^{#1}}}

%% Statistics
\newcommand{\average}[1]{\ensuremath{\left\langle #1\right\rangle}\xspace}

%% in the order of
\newcommand{\order}[1]{\ensuremath{\mathcal{O}\left(#1\right)}}

%% Vector in accordance with ISO 80000-2
\newcommand{\vect}[1]{\ensuremath\mathbfit{#1}\xspace}

%% Tensor/matrix in accordance with ISO 80000-2
\newcommand{\mtrx}[1]{\ensuremath\mathbfit{#1}\xspace}

%% Absolute
\newcommand{\abs}[1]{\ensuremath\left\lvert#1\right\rvert\xspace}
\newcommand{\absinline}[1]{\ensuremath\lvert#1\rvert\xspace}

%% Group theory
\newcommand{\group}[2]{\ensuremath{\text{#1}\!\left(#2\right)}}

%% real and imaginary part
% see: http://tex.stackexchange.com/questions/239900/begindocument-breaks-redefinition-of-re
\AtBeginDocument{\renewcommand{\Re}{\operatorname{Re}}}
\AtBeginDocument{\renewcommand{\Im}{\operatorname{Im}}}

%% imaginary unit
\newmathsymbol{\ii}{\mathrm{i}}

%% unity element 1
\newmathsymbol{\unity}{\mathbb{1}}

%% Kronecker delta (ISO 80000-2)
\newcommand{\kronecker}[1] {\delta_{#1}}

%% Column vector
%% Usage: \colvec{3}{d}{s}{b}
\newcount\colveccount
\newcommand*\colvec[1]{
        \global\colveccount#1
        \begin{pmatrix}
        \colvecnext
}
\def\colvecnext#1{
        #1
        \global\advance\colveccount-1
        \ifnum\colveccount>0
                \\
                \expandafter\colvecnext
        \else
                \end{pmatrix}
        \fi
}
