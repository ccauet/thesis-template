%!TEX root = ../../main.tex

% To-do notes
\usepackage[colorinlistoftodos]{todonotes}
% Modified listoftodos and todo commands to include a link pointing back to the
% list of todos for every todo item
\makeatletter
  \renewcommand{\listoftodos}[1][\@todonotes@todolistname]{%
    \@ifundefined{chapter}{%
      \protect{\hypertarget{todonoteslist}{\section*{#1}}}}
    {%
      \protect{\hypertarget{todonoteslist}{\chapter*{#1}}}
    }%
    \@starttoc{tdo}%
  }
\makeatother
\LetLtxMacro{\temptodo}{\todo}
\renewcommand{\todo}[2][]{%
  \temptodo[caption={#2}, bordercolor=lightgray, #1]{#2 \hfill \hyperlink{todonoteslist}{\ensuremath{\uparrow}}}%
}
% predefined todonote commands
% important todo, bold and red
\newcommand{\important}[1]{\todo[color=red, bordercolor=lightgray]{\textbf{#1}}}
% just to inform, lime and italic
\newcommand{\info}[1]{\todo[color=lime!80, bordercolor=lightgray]{\textit{#1}}}
% mark missing reference
\newcommand{\addref}[1]{\todo[color=cyan, bordercolor=lightgray, noinline]{\textit{Add Ref:} #1}}
% highlight text and add note
\usepackage{soul}
\newcommand{\replace}[2]{\texthl{#1}\xspace\todo[color=yellow, bordercolor=lightgray, noinline]{\textit{Replace by:} #2}}

% add the following lines to your private config if desired:
% \presetkeys{todonotes}{disable}{}      to disable todonotes
% \presetkeys{todonotes}{inline}{}       to have 'inline' option as default


% Blindtext
\usepackage[pangram]{blindtext}

% Linenumbers
% add \linenumbers after \begin{document} to activate line numbering
\usepackage{lineno}

% fix to allow peaceful coexistence of line numbering and
% mathematical objects
% http://www.latex-community.org/forum/viewtopic.php?f=5&t=163
\newcommand*\patchAmsMathEnvironmentForLineno[1]{%
\expandafter\let\csname old#1\expandafter\endcsname\csname #1\endcsname
\expandafter\let\csname oldend#1\expandafter\endcsname\csname
end#1\endcsname
 \renewenvironment{#1}%
   {\linenomath\csname old#1\endcsname}%
   {\csname oldend#1\endcsname\endlinenomath}%
}
\newcommand*\patchBothAmsMathEnvironmentsForLineno[1]{%
  \patchAmsMathEnvironmentForLineno{#1}%
  \patchAmsMathEnvironmentForLineno{#1*}%
}
\AtBeginDocument{%
\patchBothAmsMathEnvironmentsForLineno{equation}%
\patchBothAmsMathEnvironmentsForLineno{align}%
\patchBothAmsMathEnvironmentsForLineno{flalign}%
\patchBothAmsMathEnvironmentsForLineno{alignat}%
\patchBothAmsMathEnvironmentsForLineno{gather}%
\patchBothAmsMathEnvironmentsForLineno{multline}%
}